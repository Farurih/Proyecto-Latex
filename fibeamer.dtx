% \iffalse\begin{macrocode}
%<*driver>

\documentclass{ltxdoc}
\usepackage[utf8]{inputenc} % this file uses UTF-8
\usepackage[english]{babel}
\usepackage{tgpagella}
\usepackage{tabularx}
\usepackage{hologo}
\usepackage{booktabs}
\usepackage[scaled=0.86]{berasans}
\usepackage[scaled=1.03]{inconsolata}
\usepackage[resetfonts]{cmap}
\usepackage[T1]{fontenc} % use 8bit fonts
\emergencystretch 2dd
\usepackage{hypdoc}
\usepackage[protrusion]{microtype}
\usepackage{ragged2e}

% Making paragraphs numbered
\makeatletter
\renewcommand\paragraph{\@startsection{paragraph}{4}{\z@}%
            {-2.5ex\@plus -1ex \@minus -.25ex}%
            {1.25ex \@plus .25ex}%
            {\normalfont\normalsize\bfseries}}
\makeatother
\setcounter{secnumdepth}{4} % how many sectioning levels to assign
\setcounter{tocdepth}{4}    % how many sectioning levels to show

% ltxdoc class options
\CodelineIndex
\MakeShortVerb{|}
\EnableCrossrefs
\DoNotIndex{}
\makeatletter
\c@IndexColumns=2
\makeatother

\begin{document}
  \RecordChanges
  \DocInput{fibeamer.dtx}
  \PrintIndex
  \RaggedRight
  \PrintChanges
\end{document}

%</driver>
%    \end{macrocode}
%<*class>
\NeedsTeXFormat{LaTeX2e}
% Define `\fibeamer@version` and store it in the `VERSION.tex` file \fi
\def\fibeamer@version{2015/11/18 v1.0.2 fibeamer MU beamer theme}
% {\newwrite\f\openout\f=VERSION\write\f{\fibeamer@version}\closeout\f}
%
%%%%%%%%%%%%%%%%%%%%%%%%%%%%%%%%%%%%%%%%%%%%%%%%%%%%%%%%%%%%%%%%%%%%%%%%%%%%%%%
%
% \title{The beamer theme for the typesetting of thesis defense
%   presentations at the Masaryk University in Brno}
% \author{Vít Novotný}
% \date{\today}
% \maketitle
%
% \begin{abstract}
% \noindent This document details the design and the implementation
% of the \textsf{fibeamer} theme for the \textsf{beamer} document
% class. Included are technical information for anyone who wishes
% to extend the theme with their own color themes as well as
% information for ordinary users.
% \end{abstract}
%
% \tableofcontents
%
% \section{Basic usage}
% In order to use the \textsf{fibeamer} theme, insert
% |\usetheme|\oarg{options}|{fibeamer}| into the preamble of a
% \LaTeX\ document that uses the \textsf{beamer} document class.
% Refer to Section \ref{sec:options} for the list of available
% \textit{options}.
% 
% \section{Package options}\label{sec:options}
% \subsection{The \texttt{fonts} option}
% \begin{macro}{\iffibeamer@fonts}
% The |fonts| option instructs the package to set up the
% combination of the font families of Carlito, Arev, Iwona, Dsfont
% and DejaVu Sans Mono for the typesetting of roman, italic or
% monospaced text and mathematics. This option is enabled by
% default.
%    \begin{macrocode}
\ProvidesPackage{fibeamer/beamerthemefibeamer}[\fibeamer@version]
\newif\iffibeamer@fonts
\DeclareOptionBeamer{fonts}{\fibeamer@fontstrue}
\ExecuteOptionsBeamer{fonts}
%    \end{macrocode}
% \end{macro}
% \subsection{The \texttt{nofonts} option}
% The |nofonts| option instructs the package not to alter the
% currently set text and mathematics font families.
%    \begin{macrocode}
\DeclareOptionBeamer{nofonts}{\fibeamer@fontsfalse}
%    \end{macrocode}
% \subsection{The \texttt{microtype} option}
% \changes{v1.0.2}{2015/11/18}{Added the opt-out \texttt{microtype
% option}. [VN]}
% \begin{macro}{\iffibeamer@microtype}
% The |microtype| option instructs the package to use the
% microtypographic extensions of modern \TeX\ engines, such as
% \hologo{pdfTeX}, \Hologo{LuaTeX}, and (partially)
% \Hologo{XeLaTeX}. This option is enabled by default.
%    \begin{macrocode}
\newif\iffibeamer@microtype
\DeclareOptionBeamer{microtype}{\fibeamer@microtypetrue}
\ExecuteOptionsBeamer{microtype}
%    \end{macrocode}
% \end{macro}
% \subsection{The \texttt{nomicrotype} option}
% The |nomicrotype| option disables the microtypographic
% extensions. This may be necessary, if an older \TeX\ engine,
% such as \hologo{TeX} or \hologo{eTeX}, is being used.
%    \begin{macrocode}
\DeclareOptionBeamer{nomicrotype}{\fibeamer@microtypefalse}
%    \end{macrocode}
% \begin{macro}{\fibeamer@university}
% \subsection{The \texttt{university} option}
% The \marg{\texttt{university}=identifier} option pair sets the
% identifier of the university, at which the presentation is being
% written, to \textit{identifier}. The \textit{identifier} is
% stored within the |\fibeamer@university| macro, whose
% implicit value is \texttt{mu}. This value corresponds to the
% Masaryk University in Brno.
%    \begin{macrocode}
\DeclareOptionBeamer{university}{\def\fibeamer@university{#1}}
\ExecuteOptionsBeamer{university=mu}
%    \end{macrocode}
% \end{macro}
% \begin{macro}{\fibeamer@faculty}
% \subsection{The \texttt{faculty} option}
% The \marg{\texttt{faculty}=identifier} pair sets the faculty, at
% which the thesis is being written, to \textit{domain}. The
% following faculty \textit{identifier}s are recognized at the
% Masaryk University in Brno:
% \begin{center}\begin{tabularx}{\textwidth}{Xc}\toprule
%   The faculty & The \textit{domain} name \\\midrule
%   The Faculty of Informatics & \texttt{fi} \\
%   The Faculty of Science & \texttt{sci} \\
%   The Faculty of Law & \texttt{law} \\
%   The Faculty of Economics and Administration & \texttt{econ} \\
%   The Faculty of Social Studies & \texttt{fss} \\
%   The Faculty of Medicine & \texttt{med} \\
%   The Faculty of Education & \texttt{ped} \\
%   The Faculty of Arts & \texttt{phil} \\
%   The Faculty of Sports Studies & \texttt{fsps} \\\bottomrule
% \end{tabularx}\end{center}
% The \textit{identifier} is stored within the |\fibeamer@faculty|
% macro, whose implicit value is \texttt{fi}.
%    \begin{macrocode}
\DeclareOptionBeamer{faculty}{\def\fibeamer@faculty{#1}}
\ExecuteOptionsBeamer{faculty=fi}
%    \end{macrocode}
% \end{macro}
% \begin{macro}{\fibeamer@basePath}
% The \marg{\texttt{basePath}=path} pair sets the \textit{path}
% containing the package files. The \textit{path} is prepended to
% every other path (|\fibeamer@logopath| and |\fibeamer@themePath|)
% used by the package. If non-empty, the
% \textit{path} gets normalized to \textit{path/}. The normalized
% \textit{path} is stored within the |\fibeamer@basePath| macro,
% whose implicit value is |fibeamer/|.
%    \begin{macrocode}
\DeclareOptionBeamer{basePath}{%
  \ifx\fibeamer@empty#1\fibeamer@empty%
    \def\fibeamer@basePath{}%
  \else%
    \def\fibeamer@basePath{#1/}%
  \fi}
\ExecuteOptionsBeamer{basePath=fibeamer}
%    \end{macrocode}
% \end{macro}
% \begin{macro}{\fibeamer@subdir}
% The |\fibeamer@subdir| macro returns |/| unchanged, coerces
% |.|, |..|, |/|\textit{path}, |./|\textit{path} and
% |../|\textit{path} to |./|, |../|, |/|\textit{path}|/|,
% |./|\textit{path}|/| and |../|\textit{path}|/|, respectively, and
% prefixes any other \textit{path} with |\fibeamer@basePath|. This
% macro is used within the definition of the \texttt{themePath} and
% \texttt{logoPath} options.
%    \begin{macrocode}
\def\fibeamer@subdir#1#2#3#4\empty{%
  \ifx#1\empty%           <empty> -> <basePath>
    \fibeamer@basePath
  \else
    \if#1/%
      \ifx#2\empty%             / -> /
        /%
      \else%              /<path> -> /<path>/
        #1#2#3#4/%
      \fi
    \else%
      \if#1.%
        \ifx#2\empty%           . -> ./
          ./%
        \else
          \if#2.%
            \ifx#3\empty%      .. -> ../
              ../%
            \else
              \if#3/%   ../<path> -> ../<path>/
                ../#4/%
              \else
                \fibeamer@basePath#1#2#3#4/%
              \fi
            \fi
          \else
            \if#2/%      ./<path> -> ./<path>/
              ./#3#4/%
            \else
              \fibeamer@basePath#1#2#3#4/%
            \fi
          \fi
        \fi
      \else
        \fibeamer@basePath#1#2#3#4/%
      \fi
    \fi%
  \fi}
%    \end{macrocode}
% \end{macro}
% \begin{macro}{\fibeamer@themePath}
% \subsection{The \texttt{themePath} option}
% The \marg{\texttt{themePath}=path} pair sets the \textit{path}
% containing the theme files. The \textit{path} is normalized using
% the |\fibeamer@subdir| macro and stored within the
% |\fibeamer@stylePath| macro, whose implicit value is
% |\fibeamer@basepath theme/|. By default, this expands to
% \texttt{fibeamer/theme/}.
%    \begin{macrocode}
\DeclareOptionBeamer{themePath}{%
  \def\fibeamer@themePath{\fibeamer@subdir#1%
    \empty\empty\empty\empty}}
\ExecuteOptionsBeamer{themePath=theme}
%    \end{macrocode}
% \end{macro}
% \begin{macro}{\fibeamer@logoPath}
% \subsection{The \texttt{logoPath} option}
% The \marg{\texttt{logoPath}=path} pair sets the \textit{path}
% containing the logo files, which is used by the outer themes to
% load the faculty logos. The \textit{path} is
% normalized using the |\fibeamer@subdir| macro and stored
% within the |\fibeamer@logoPath| macro, whose implicit value
% is |\fibeamer@basePath| followed by |logo/\fibeamer@university/|. By
% default, this expands to \texttt{fibeamer/logo/mu/}.
%    \begin{macrocode}
\DeclareOptionBeamer{logoPath}{%
  \def\fibeamer@logoPath{\fibeamer@subdir#1%
    \empty\empty\empty\empty}}
\ExecuteOptionsBeamer{logoPath=logo/\fibeamer@university}
%    \end{macrocode}
% \end{macro}
% \begin{macro}{\fibeamer@logo}
% \subsection{The \texttt{logo} option}
% The \marg{\texttt{logo}=filename} pair sets the prefix of the
% filename of the logo file to be used as the faculty logo to
% \textit{filename}. The \textit{filename} is stored within the
% |\fibeamer@logo| macro, whose implicit value is
% |\fibeamer@logoPath| \texttt{fibeamer-}|\fibeamer@university|%
% \texttt{-}|\fibeamer@faculty|. By default, this expands to
% \texttt{fibeamer/logo/mu/fibeamer-mu-fi}. The filenames of the
% actual files are |\fibeamer@logo|-light and |\fibeamer@logo|-dark,
% and correspond to the light and dark versions of the given logo.
%    \begin{macrocode}
\DeclareOptionBeamer{logo}{\def\fibeamer@logo{#1}}
\ExecuteOptionsBeamer{%
  logo=\fibeamer@logoPath fibeamer-\fibeamer@university-\fibeamer@faculty}
%    \end{macrocode}
% \end{macro}
% \section{Logic}
% \subsection{Macros}
% \begin{macro}{\fibeamer@require}
% The |\fibeamer@require|\marg{package} macro is used to gracefully
% load a \textit{package}. Packages that have already been loaded
% or do not exist are ignored by the macro.
%    \begin{macrocode}
\def\fibeamer@require#1{\IfFileExists{#1.sty}{%
  \@ifpackageloaded{#1}{}{\RequirePackage{#1}}}{}}
%    \end{macrocode}
% \end{macro}
% \begin{macro}{\fibeamer@requireTheme}
% The |\fibeamer@requireTheme|\marg{class} macro is used to load a
% \textit{class} of beamer themes. The following packages are
% loaded, provided they exist:
% \begin{itemize}
%  \item\texttt{beamer}\textit{class}\texttt{themefibeamer} from
%    the |\fibeamer@themePath| directory -- The base theme file. By
%    default, this expands to \texttt{fibeamer/theme/beamer}^^A
%    \textit{class}\texttt{themefibeamer}.
%  \item\texttt{beamer}\textit{class}\texttt{themefibeamer-}^^A
%    |\fibeamer@university| from the |\fibeamer@themePath|^^A
%    \discretionary{}{}{}|\fibeamer@university/| directory -- The
%    university theme file. By default, this expands to
%    \texttt{fibeamer/theme/mu/beamer}\textit{class}^^A
%    \texttt{themefibeamer-mu}.
%  \item\texttt{beamer}\textit{class}\texttt{themefibeamer-}^^A
%    |\fibeamer@university|\texttt{-}|\fibeamer@faculty| from the
%    |\fibeamer@themePath\fibeamer@university/| directory -- The
%    faculty theme file. By default, this expands to \texttt{^^A
%    fibeamer/theme/mu/beamer}\textit{class}\texttt{themefibeam^^A
%    er-mu-fi}.
% \end{itemize}
%    \begin{macrocode}
\def\fibeamer@requireTheme#1{%
  \fibeamer@require{\fibeamer@themePath beamer#1themefibeamer}
  \fibeamer@require{\fibeamer@themePath\fibeamer@university%
    /beamer#1themefibeamer-\fibeamer@university}
  \fibeamer@require{\fibeamer@themePath\fibeamer@university%
    /beamer#1themefibeamer-\fibeamer@university-\fibeamer@faculty}}
%    \end{macrocode}
% \end{macro}
% \subsection{Main routine}
% First, the theme processes the options.
%    \begin{macrocode}
\ProcessOptionsBeamer
%    \end{macrocode}
% When the |microtype| option is specified, the \textsf{microtype}
% package gets loaded.
%    \begin{macrocode}
  % Set up the microtypographic extensions
  \iffibeamer@microtype
    \RequirePackage{microtype}
  \fi
%    \end{macrocode}
% When the |fonts| option is specified, the following packages will
% be used by the package to configure the fonts in the presentation
% mode:
% \begin{itemize}
%   \item\textsf{ifthen} -- This package is used to construct
%     compound conditionals.
%   \item\textsf{ifxetex}, \textsf{ifluatex} -- These packages are
%     used to detect the used \TeX\ engine.
%   \item\textsf{lmodern} -- The Latin Modern font family is used
%     as a fallback for missing glyphs.
%   \item\textsf{carlito} -- The Carlito font family is used as the
%     primary text and math font face.
%   \item\textsf{arevmath} -- The Arev math font family is used for
%     various symbols and greek alphabet.
%   \item\textsf{iwona} -- The Iwona font family is used for large
%     mathAsymbols.
%   \item\textsf{DejaVuSansMono} -- The DejaVu Sans Mono font
%     family is used for the typesetting of monospaced text.
%   \item\textsf{setspace} -- This package is used to adjust the
%     leading to 115 \%.
%   \item\textsf{fontenc} -- This package is used to set the font
%     encoding to Cork. This package is only used outside the
%     \Hologo{XeTeX} and \Hologo{LuaTeX} engines.
%   \item\textsf{fontenc} -- This package is used to load fonts.
%     This package is only used with the \Hologo{XeTeX} and
%     \Hologo{LuaTeX} engines.
% \end{itemize}
%    \begin{macrocode}
\mode<presentation>
  % Set up the fonts
  \iffibeamer@fonts
    \RequirePackage{ifthen}
    \RequirePackage{ifxetex}
    \RequirePackage{ifluatex}
    \RequirePackage{lmodern}
    \RequirePackage[sfdefault,lf]{carlito}
    \renewcommand*\oldstylenums[1]{{\carlitoOsF #1}}

    %% Load arev with scaling factor of .85
    %% See <http://tex.stackexchange.com/a/181240/70941>
    \DeclareFontFamily{OML}{zavm}{\skewchar\font=127 }
    \DeclareFontShape{OML}{zavm}{m}{it}{<-> s*[.85] zavmri7m}{}
    \DeclareFontShape{OML}{zavm}{b}{it}{<-> s*[.85] zavmbi7m}{}
    \DeclareFontShape{OML}{zavm}{m}{sl}{<->ssub * zavm/m/it}{}
    \DeclareFontShape{OML}{zavm}{bx}{it}{<->ssub * zavm/b/it}{}
    \DeclareFontShape{OML}{zavm}{b}{sl}{<->ssub * zavm/b/it}{}
    \DeclareFontShape{OML}{zavm}{bx}{sl}{<->ssub * zavm/b/sl}{}

    \AtBeginDocument{
      \SetSymbolFont{operators}   {normal}{OT1}{zavm}{m}{n}
      \SetSymbolFont{letters}     {normal}{OML}{zavm}{m}{it}
      \SetSymbolFont{symbols}     {normal}{OMS}{zavm}{m}{n}
      \SetSymbolFont{largesymbols}{normal}{OMX}{iwona}{m}{n}}
    \RequirePackage[sans]{dsfont}

    \ifthenelse{\boolean{xetex}\OR\boolean{luatex}}{
      \RequirePackage{fontspec}
      \setmonofont[Scale=0.85,Ligatures=TeX]{DejaVu Sans Mono}
    }{
      \usepackage[scaled=0.85]{DejaVuSansMono}
      \RequirePackage[resetfonts]{cmap}
      \RequirePackage[T1]{fontenc}
    }
    \RequirePackage{setspace}
    \setstretch{1.15}
  \fi
\mode
  <all>
%    \end{macrocode}
% Finally, the color, font, inner and outer themes of the
% respective university and faculty will be loaded.
%    \begin{macrocode}
\fibeamer@requireTheme{color}
\fibeamer@requireTheme{font}
\fibeamer@requireTheme{inner}
\fibeamer@requireTheme{outer}
%    \end{macrocode}
%
% \section{Themes}
% This section contains the combined documentation of all available
% themes.  When creating a new theme file, it is advisable to
% create one self-contained \texttt{dtx} file, which is then
% partitioned into locale files via the \textsf{docstrip} tool
% based on the respective \texttt{ins} file. A
% \DescribeMacro{\file} macro |\file|\marg{filename} is available
% for the sectioning of the documentation of various files within
% the \texttt{dtx} file. For more information about \texttt{dtx}
% files and the \textsf{docstrip} tool, consult the \textsf{dtxtut,
% docstrip, doc} and \textsf{ltxdoc} manuals.
%
% \def\file#1{\subsubsection{The \texttt{#1} file}}
%
% \subsection{Base files}
% % \iffalse
%<*color>
% \fi\file{theme/mu/beamercolorthemefibeamer-mu.sty}
% This is the base color theme for presentations written at the
% Masaryk University in Brno.
%    \begin{macrocode}
\NeedsTeXFormat{LaTeX2e}
\ProvidesPackage{fibeamer/theme/mu/%
  beamercolorthemefibeamer-mu}[2016/01/12]
%    \end{macrocode}
% \begin{macro}{\darkframes}
% The |darkframes| environment switches the color definitions to
% render the enclosed frames in dark colors. This is a dummy
% definition, which will be overridden by the subsequently loaded
% color theme in the presentation mode.
%    \begin{macrocode}
\newenvironment{darkframes}{}{}
%    \end{macrocode}
% \end{macro}
% The rest of the theme will be ignored outside the presentation
% mode.
%    \begin{macrocode}
\mode<presentation>
%    \end{macrocode}
% The theme loads the following packages, which will be used by the
% subsequently loaded color theme specific to a faculty:
% \begin{itemize}
%   \item\textsf{listings} -- This package is used for code
%     listings. The subsequently loaded color theme will specify
%     source code coloring for the package.
%   \item\textsf{ifthen} -- This package is used to construct
%     compound conditionals.
%   \item\textsf{tikz} -- This package is used to create gradient
%     background for dark slides.
% \end{itemize}
%    \begin{macrocode}
  \RequirePackage{listings}
  \RequirePackage{ifthen}
  \RequirePackage{tikz}
%    \end{macrocode}
% \begin{macro}{\iffibeamer@dark}
% The |\iffibeamer@dark| conditional will be switched on and off by
% the subsequently loaded color theme based on whether or not the
% given frame is being typeset in light or dark colors. This
% information will be used by outer themes to insert the correct
% logo into each frame.
%    \begin{macrocode}
  \newif\iffibeamer@dark\fibeamer@darkfalse
%    \end{macrocode}
% \end{macro}
% A frame that is either title or dark, as specified by the value
% of the |\iffibeamer@dark| conditional, will have a gradient
% background. The |fibeamer@backgroundInner| and
% |fibeamer@backgroundOuter| colors will be defined by the
% subsequently loaded color theme.
%    \begin{macrocode}
\defbeamertemplate*{background canvas}{fibeamer}{%
  \ifthenelse{%
    \boolean{fibeamer@dark} \OR \c@framenumber=0
  }{%
    \begin{tikzpicture}
      \clip (0,\fibeamer@lengths@clipbottom) rectangle
        (\paperwidth,\fibeamer@lengths@cliptop);
      \path [inner color = fibeamer@backgroundInner,
             outer color = fibeamer@backgroundOuter]
        (0,0) rectangle (\paperwidth,\paperwidth);
    \end{tikzpicture}
  }{}}
\mode
<all>
%    \end{macrocode}
% \iffalse
%</color>
%<*font>
% \fi\file{theme/mu/beamerfontthemefibeamer-mu.sty}
% This is the base font theme for presentations written at the
% Masaryk University in Brno. The theme has no effect outside the
% presentation mode.
%    \begin{macrocode}
\NeedsTeXFormat{LaTeX2e}
\ProvidesPackage{fibeamer/theme/mu/%
  beamerfontthemefibeamer-mu}[2016/01/12]
\mode<presentation>
  \setbeamerfont{normal text}{size=\normalsize}
  \setbeamerfont{title}{size=\LARGE, series=\bfseries}
  \setbeamerfont{subtitle}{parent=normal text, size=\Large}
  \setbeamerfont{frametitle}{size=\Large}
  \setbeamerfont{framesubtitle}{size=\large, shape=\itshape}
  \setbeamerfont{description item}{series=\bfseries}
  \setbeamerfont{author}{size=\large}
\mode
<all>
%    \end{macrocode}
% \iffalse
%</font>
%<*inner>
% \fi\file{theme/mu/beamerinnerthemefibeamer-mu.sty}
% This is the base inner theme for presentations written at the
% Masaryk University in Brno. The theme has no effect outside the
% presentation mode.
%    \begin{macrocode}
\NeedsTeXFormat{LaTeX2e}
\ProvidesPackage{fibeamer/theme/mu/%
  beamerinnerthemefibeamer-mu}[2016/01/12]
\mode<presentation>
\defbeamertemplate*{itemize item}{fibeamer}{$\bullet$}
\defbeamertemplate*{itemize subitem}{fibeamer}{\---}
\defbeamertemplate*{itemize subsubitem}{fibeamer}{\guillemotright}
\setbeamertemplate{bibliography item}{\insertbiblabel}
%    \end{macrocode}
% \changes{v1.1.0:6}{2016/01/11}{Added support for the
%   \cs{tableofcontents}. [VN]}
%    \begin{macrocode}
\setbeamertemplate{section in toc}{%
  \usebeamercolor[fg]{item}%
    \inserttocsectionnumber.%
  \usebeamercolor[fg]{structure}%
  \kern1.25ex\inserttocsection\par}
\setbeamertemplate{subsection in toc}{%
  \hspace\leftmargini
  \usebeamercolor[fg]{item}%
    \inserttocsectionnumber.\inserttocsubsectionnumber%
  \usebeamercolor[fg]{structure}%
  \kern1.25ex\inserttocsubsection\par}
\setbeamertemplate{subsubsection in toc}{%
  \hspace\leftmargini
  \hspace\leftmarginii
  \usebeamercolor[fg]{item}%
    \inserttocsectionnumber.\inserttocsubsectionnumber.%
    \inserttocsubsubsectionnumber%
  \usebeamercolor[fg]{structure}%
  \kern1.25ex\inserttocsubsubsection\par}
\mode
<all>
%    \end{macrocode}
% \iffalse
%</inner>
%<*outer>
% \fi\file{theme/mu/beamerouterthemefibeamer-mu.sty}
% This is the base outer theme for presentations written at the
% Masaryk University in Brno. The theme has no effect outside the
% presentation mode.
%    \begin{macrocode}
\NeedsTeXFormat{LaTeX2e}
\ProvidesPackage{fibeamer/theme/mu/%
  beamerouterthemefibeamer-mu}[2016/01/12]
\mode<presentation>
%    \end{macrocode}
% The theme uses the following packages:
% \begin{itemize}
%   \item\textsf{ifthen} -- This package is used to construct
%     compound conditionals.
%   \item\textsf{ifpdf} -- This package is used to check, whether
%     the document is being typeset in DVI mode. If it is, then
%     the |\pdfpagewidth| and |\pdfpageheight| dimensions are
%     defined, so that positioning in TikZ works correctly.
%     \changes{v1.0.1}{2015/10/03}{Added DVI output support. [VN]}
%     ^^A <http://tex.stackexchange.com/a/246631/70941>
%   \item\textsf{tikz} -- This package is used to position the
%     logo and the frame number on a frame.
%   \item\textsf{pgfcore} -- This package is used to draw the
%     dashed line at the title frame.
% \end{itemize}
%    \begin{macrocode}
  \RequirePackage{ifthen}
  \RequirePackage{ifpdf}
  \ifpdf\else
    \@ifundefined{pdfpagewidth}{\newdimen\pdfpagewidth}{}
    \@ifundefined{pdfpageheight}{\newdimen\pdfpageheight}{}
    \pdfpagewidth=\paperwidth
    \pdfpageheight=\paperheight
  \fi
  \RequirePackage{tikz}
  \RequirePackage{pgfcore}
%    \end{macrocode}
% This part of the outer theme defines the geometry of the frames
% along with other dimensions.
% \changes{v1.1.0:4}{2016/01/11}{Length definitions within the
%   themes of the Masaryk University in Brno are no longer based on
%   the dimensions of the (now unused) logo in the upper right
%   corner. [VN]}
%    \begin{macrocode}
  \newlength\fibeamer@lengths@baseunit
  \fibeamer@lengths@baseunit=3.75mm
  % The footer padding
  \newlength\fibeamer@lengths@footerpad
  \setlength\fibeamer@lengths@footerpad{%
    \fibeamer@lengths@baseunit}
  % The side margins
  \newlength\fibeamer@lengths@margin
  \setlength\fibeamer@lengths@margin{%
    3\fibeamer@lengths@baseunit}
  \setbeamersize{
    text margin left=\fibeamer@lengths@margin,
    text margin right=\fibeamer@lengths@margin}
  % The upper margin
  \newlength\fibeamer@lengths@titleline
  \setlength\fibeamer@lengths@titleline{%
    3\fibeamer@lengths@baseunit}
  % The background clipping
  \newlength\fibeamer@lengths@clipbottom
  \setlength\fibeamer@lengths@clipbottom\paperwidth
  \addtolength\fibeamer@lengths@clipbottom{-\paperheight}
  \setlength\fibeamer@lengths@clipbottom{%
    0.5\fibeamer@lengths@clipbottom}
  \newlength\fibeamer@lengths@cliptop
  \setlength\fibeamer@lengths@cliptop\paperwidth
  \addtolength\fibeamer@lengths@cliptop{%
    -\fibeamer@lengths@clipbottom}
%    \end{macrocode}
% \changes{v1.1.0:6}{2016/01/11}{Added the logo dimension
%   definitions. [VN]}
%    \begin{macrocode}
  % The logo size
  \newlength\fibeamer@lengths@logowidth
  \setlength\fibeamer@lengths@logowidth{%
    14\fibeamer@lengths@baseunit}
  \newlength\fibeamer@lengths@logoheight
  \setlength\fibeamer@lengths@logoheight{%
    0.4\fibeamer@lengths@logowidth}
%    \end{macrocode}
% The outer theme completely culls the bottom navigation.
%    \begin{macrocode}
  \defbeamertemplate*{navigation symbols}{fibeamer}{}
%    \end{macrocode}
% The outer theme also culls the headline.
% \changes{v1.1.0:1}{2015/11/24}{The faculty logos are no longer
%   displayed on regular slides, as per the new unified design of
%   the Masaryk University in Brno. [VN]}
%    \begin{macrocode}
  \defbeamertemplate*{headline}{fibeamer}{}
%    \end{macrocode}
% The frame title.
%    \begin{macrocode}
  \defbeamertemplate*{frametitle}{fibeamer}{%
    \vskip-1em % Align the text with the top border
    \vskip\fibeamer@lengths@titleline
    \usebeamercolor[fg]{frametitle}%
    \usebeamerfont{frametitle}%
      \insertframetitle\par%
    \usebeamercolor[fg]{framesubtitle}%
    \usebeamerfont{framesubtitle}%
      \insertframesubtitle}
%    \end{macrocode}
% The footline contains the frame number. It is flushed right.
%    \begin{macrocode}
  \defbeamertemplate*{footline}{fibeamer}{%
    \ifnum\c@framenumber=0\else%
      \begin{tikzpicture}[overlay]
        \node[anchor=south east,
          yshift=\fibeamer@lengths@footerpad,
          xshift=-\fibeamer@lengths@footerpad] at
          (current page.south east) {
            \usebeamercolor[fg]{framenumber}%
            \usebeamerfont{framenumber}%
            \insertframenumber/\inserttotalframenumber};
      \end{tikzpicture}
    \fi}
%    \end{macrocode}
% The title frame contains the |\title| vertically positioned to
% the upper $\phi^{-1}$ of the page, where $\phi$ refers to the
% golden ratio, and underlined with a dashed line. At the bottom
% of the page, there is the |\subtitle| followed by the |\author|.
%    \begin{macrocode}
  \defbeamertemplate*{title page}{fibeamer}{%
    % This is slide 0
    \setcounter{framenumber}{0}

    % Input the university logo
    \begin{tikzpicture}[
      remember picture,
      overlay,
      xshift=0.5\fibeamer@lengths@logowidth,
      yshift=0.5\fibeamer@lengths@logoheight
    ]
      \node at (0,0) {
        \fibeamer@includeLogo[
          width=\fibeamer@lengths@logowidth,
          height=\fibeamer@lengths@logoheight
        ]};
    \end{tikzpicture}

    % Input the title
    \usebeamerfont{title}%
    \usebeamercolor[fg]{title}%
    \begin{minipage}[b][2\baselineskip][b]{\textwidth}%
      \raggedright%
      \inserttitle%
    \end{minipage}
    \vskip-.5\baselineskip

    % Input the dashed line
    \begin{pgfpicture}
      \pgfsetlinewidth{2pt}
      \pgfsetroundcap
      \pgfsetdash{{0pt}{4pt}}{0cm}

      \pgfpathmoveto{\pgfpoint{0mm}{0mm}}
      \pgfpathlineto{\pgfpoint{\textwidth}{0mm}}

      \pgfusepath{stroke}
    \end{pgfpicture}
    \vfill
    
    % Input the subtitle
    \usebeamerfont{subtitle}%
    \begin{minipage}{\textwidth}
      \raggedright%
      \insertsubtitle%
    \end{minipage}\vskip.25\baselineskip

    % Input the author's name
    \usebeamerfont{author}%
    \begin{minipage}{\textwidth}
      \raggedright%
      \insertauthor%
    \end{minipage}}

\mode
<all>
%    \end{macrocode}
% \iffalse
%</outer>
% \fi

% \subsection{The Faculty of Informatics}
% % \file{theme/mu/beamercolorthemefibeamer-fi.sty}
% This is the color theme for presentations written at the
% Faculty of Informatics at the Masaryk University in Brno.
% This theme has no effect outside the presentation mode.
%    \begin{macrocode}
\NeedsTeXFormat{LaTeX2e}
\ProvidesPackage{fibeamer/theme/mu/%
  beamercolorthemefibeamer-mu-fi}[2016/01/14]
\mode<presentation>
%    \end{macrocode}
% This color theme uses the combination of yellow and dark gray in light frames
% and the combination of gold and white in dark frames.  The
% |fibeamer@back|\-|groundInner| and |fibeamer@backgroundOuter| colors are used
% within the background canvas template, which is defined within the color
% theme of the Masaryk University and which draws the gradient background of
% dark frames.
%    \begin{macrocode}
  \definecolor{fibeamer@gold}{HTML}{a47312}
  \definecolor{fibeamer@yellow}{HTML}{FFEB9C}
  \definecolor{fibeamer@darkYellow}{HTML}{FFB600}
  \definecolor{fibeamer@orange}{HTML}{FF5500}
  \definecolor{fibeamer@gray}{HTML}{999999}
  \colorlet{fibeamer@blue}{blue!60!fibeamer@gray}
  \colorlet{fibeamer@backgroundInner}{fibeamer@gold}
  \colorlet{fibeamer@backgroundOuter}{fibeamer@gold!60!black}
%    \end{macrocode}
% The |darkframes| environment switches the |\iffibeamer@darktrue|
% conditional on and sets a dark color theme.
%    \begin{macrocode}
  \renewenvironment{darkframes}{%
    \begingroup
      \fibeamer@darktrue
      %% Structures
      \setbeamercolor*{frametitle}{fg=fibeamer@gold!30!white}
      \setbeamercolor*{framesubtitle}{fg=white}
      %% Text
      \setbeamercolor*{normal text}{fg=white, bg=fibeamer@gold}
      \setbeamercolor*{structure}{fg=white, bg=fibeamer@gold}
%    \end{macrocode}
% \changes{v1.1.0:7}{2016/01/12}{Added support for \cs{alert} to
%   the themes of the Masaryk University in Brno. [VN]}
%    \begin{macrocode}
      \setbeamercolor*{alerted text}{%
        fg=fibeamer@gold!70!white!20!fibeamer@darkYellow}
      %% Items
      \setbeamercolor*{item}{fg=fibeamer@gold!30!white}
      \setbeamercolor*{footnote mark}{fg=fibeamer@gold!30!white}
      %% Blocks
      \setbeamercolor*{block title}{%
        fg=white, bg=fibeamer@gold!60!white}
      \setbeamercolor*{block title example}{%
        fg=white, bg=fibeamer@gold!60!white}
      \setbeamercolor*{block title alerted}{%
        fg=white, bg=fibeamer@gold!90!white}
      \setbeamercolor*{block body}{%
        fg=fibeamer@gold, 
        bg=fibeamer@gray!15!white}
      \usebeamercolor*{normal text}
      % Code listings
      \lstset{%
        commentstyle=\color{green!30!white},
        keywordstyle=\color{blue!30!white},
        stringstyle=\color{fibeamer@gold!30!white}}
      }{%
    \endgroup}
%    \end{macrocode}
% Outside the |darkframes| environment, the light theme is used.
%    \begin{macrocode}
  %% Structures
  \setbeamercolor{frametitle}{fg=black!75!white}
  \setbeamercolor{framesubtitle}{fg=black!50!white}
  %% Text
  \setbeamercolor{normal text}{fg=black!75!white, bg=white}
  \setbeamercolor{structure}{fg=black!75!white, bg=white}
%    \end{macrocode}
% \changes{v1.1.0:7}{2016/01/12}{Added support for \cs{alert} to
%   the themes of the Masaryk University in Brno. [VN]}
%    \begin{macrocode}
  \setbeamercolor{alerted text}{fg=fibeamer@orange}
  \addtobeamertemplate{block begin}{%
    \iffibeamer@dark % alerted text in plain block at dark slides
      \setbeamercolor{alerted text}{%
        fg=fibeamer@gold!30!fibeamer@darkYellow}%
    \fi}{}
  %% Items
  \setbeamercolor{item}{fg=fibeamer@darkYellow}
  \setbeamercolor{footnote mark}{fg=fibeamer@darkYellow}
  %% Blocks
  \setbeamercolor{block title}{%
    fg=black!75!white, bg=fibeamer@yellow}
  \setbeamercolor{block title example}{%
    fg=black!75!white, bg=fibeamer@yellow}
  \setbeamercolor{block title alerted}{%
    fg=white, bg=black!75!white}
  \setbeamercolor{block body}{%
    fg=fibeamer@yellow, bg=black!75!white}
  %% Title
  \setbeamercolor{title}{fg=white, bg=fibeamer@gold}
  % Code listings
  \lstset{%
    basicstyle=\footnotesize\ttfamily,
    breakatwhitespace=false,
    breaklines=true,
    commentstyle=\color{green!60!black},
    extendedchars=true,
    keywordstyle=\color{fibeamer@blue},
    showspaces=false,
    showstringspaces=false,
    showtabs=false,
    stringstyle=\color{violet}}
\mode
<all>
%    \end{macrocode}

% \subsection{The Faculty of Science}
% % \file{theme/mu/beamercolorthemefibeamer-sci.sty}
% This is the color theme for presentations written at the Faculty
% of Science at the Masaryk University in Brno. This theme has no
% effect outside the presentation mode.
%    \begin{macrocode}
\NeedsTeXFormat{LaTeX2e}
\ProvidesPackage{fibeamer/theme/mu/%
  beamercolorthemefibeamer-mu-sci}[2016/05/06]
\mode<presentation>
%    \end{macrocode}
% This color theme uses the combination of yellow and shades of gray.  The
% |fibeamer@{dark,|\-|light}@background{Inner,|\-|Outer}| colors are used
% within the background canvas template, which is defined within the base
% color theme of the Masaryk University and which draws the gradient
% background of the frames.
%    \begin{macrocode}
  \definecolor{fibeamer@black}{HTML}{000000}
  \definecolor{fibeamer@white}{HTML}{FFFFFF}
  \definecolor{fibeamer@green}{HTML}{139632}
  \definecolor{fibeamer@gray}{HTML}{999999}
  \colorlet{fibeamer@lightGreen}{fibeamer@green!30!fibeamer@white}
  \colorlet{fibeamer@darkGreen}{fibeamer@green!60!fibeamer@black}
  \definecolor{fibeamer@lightOrange}{HTML}{FFA25E}
  \colorlet{fibeamer@orange}{fibeamer@lightOrange!80!fibeamer@darkGreen}
%    \end{macrocode}
% \changes{v1.1.4:2}{2016/05/06}{Removed gradient backgrounds from
%   the color themes of the Masaryk University in Brno. [VN]}
%    \begin{macrocode}
  %% Background gradients
  \colorlet{fibeamer@dark@backgroundInner}{fibeamer@darkGreen}
  \colorlet{fibeamer@dark@backgroundOuter}{fibeamer@darkGreen}
  \colorlet{fibeamer@light@backgroundInner}{fibeamer@white}
  \colorlet{fibeamer@light@backgroundOuter}{fibeamer@white}
%    \end{macrocode}
% The |darkframes| environment switches the |\iffibeamer@darktrue|
% conditional on and sets a dark color theme.
%    \begin{macrocode}
  \renewenvironment{darkframes}{%
    \begingroup
      \fibeamer@darktrue
      %% Structures
      \setbeamercolor*{frametitle}{fg=fibeamer@lightGreen}
      \setbeamercolor*{framesubtitle}{fg=fibeamer@white}
      %% Text
      \setbeamercolor*{normal text}{fg=fibeamer@white, bg=fibeamer@green}
      \setbeamercolor*{structure}{fg=fibeamer@white, bg=fibeamer@green}
%    \end{macrocode}
% \changes{v1.1.0:7}{2016/01/12}{Added support for \cs{alert} to
%   the themes of the Masaryk University in Brno. [VN]}
% \changes{v1.1.4:5}{2016/06/06}{Unified the alert colors in the
%   color themes of the Masaryk University in Brno. [VN]}
%    \begin{macrocode}
      \setbeamercolor*{alerted text}{%
        fg=fibeamer@lightOrange}
%    \end{macrocode}
% \changes{v1.1.4:3}{2016/05/06}{Added proper link coloring for the
%   color themes of the Masaryk University in Brno. [VN]}
%    \begin{macrocode}
      %% Items, footnotes and links
      \setbeamercolor*{item}{fg=fibeamer@lightGreen}
      \setbeamercolor*{footnote mark}{fg=fibeamer@lightGreen}
      \hypersetup{urlcolor=fibeamer@lightGreen}
      %% Blocks
      \setbeamercolor*{block title}{%
        fg=fibeamer@white, bg=fibeamer@green!60!fibeamer@white}
      \setbeamercolor*{block title example}{%
        fg=fibeamer@white, bg=fibeamer@green!60!fibeamer@white}
      \setbeamercolor*{block title alerted}{%
        fg=fibeamer@darkGreen, bg=fibeamer@lightOrange}
      \setbeamercolor*{block body}{%
        fg=fibeamer@green, 
        bg=fibeamer@gray!15!fibeamer@white}
      \usebeamercolor*{normal text}
      % Code listings
      \lstset{%
        commentstyle=\color{green!30!fibeamer@white},
        keywordstyle=\color{blue!30!fibeamer@white},
        stringstyle=\color{red!30!fibeamer@white}}
      }{%
    \endgroup}
%    \end{macrocode}
% Outside the |darkframes| environment, the light theme is used.
%    \begin{macrocode}
  %% Structures
  \setbeamercolor{frametitle}{fg=fibeamer@green}
  \setbeamercolor{framesubtitle}{fg=fibeamer@black!75!fibeamer@white}
  %% Text
  \setbeamercolor{normal text}{fg=fibeamer@black, bg=fibeamer@white}
  \setbeamercolor{structure}{fg=fibeamer@black, bg=fibeamer@white}
%    \end{macrocode}
% \changes{v1.1.0:7}{2016/01/12}{Added support for \cs{alert} to
%   the themes of the Masaryk University in Brno. [VN]}
% \changes{v1.1.4:5}{2016/06/06}{Unified the alert colors in the
%   color themes of the Masaryk University in Brno. [VN]}
%    \begin{macrocode}
  \setbeamercolor{alerted text}{fg=red}
  \addtobeamertemplate{block begin}{%
    \iffibeamer@dark % alerted text in plain block at dark slides
      \setbeamercolor{alerted text}{fg=fibeamer@orange}%
    \fi}{}
%    \end{macrocode}
% \changes{v1.1.4:3}{2016/05/06}{Added proper link coloring for the
%   color themes of the Masaryk University in Brno. [VN]}
%    \begin{macrocode}
  %% Items, footnotes and links
  \setbeamercolor*{item}{fg=fibeamer@green}
  \setbeamercolor*{footnote mark}{fg=fibeamer@green}
  \hypersetup{urlcolor=fibeamer@green}
  %% Blocks
  \setbeamercolor{block title}{%
    fg=fibeamer@white, bg=fibeamer@green}
  \setbeamercolor{block title example}{%
    fg=fibeamer@white, bg=fibeamer@green}
  \setbeamercolor{block title alerted}{%
    fg=fibeamer@white, bg=red}
  \setbeamercolor{block body}{%
    fg=fibeamer@green, bg=fibeamer@gray!20!fibeamer@white}
  %% Title
  \setbeamercolor{title}{fg=fibeamer@white, bg=fibeamer@green}
  % Code listings
  \lstset{%
    basicstyle=\footnotesize\ttfamily,
    breakatwhitespace=false,
    breaklines=true,
    commentstyle=\color{green!60!fibeamer@black},
    extendedchars=true,
    keywordstyle=\color{blue},
    showspaces=false,
    showstringspaces=false,
    showtabs=false,
    stringstyle=\color{violet}}
\mode
<all>
%    \end{macrocode}

% \subsection{The Faculty of Arts}
% % \file{theme/mu/beamercolorthemefibeamer-phil.sty}
% This is the color theme for presentations written at the Faculty
% of Arts at the Masaryk University in Brno. This theme has no
% effect outside the presentation mode.
%    \begin{macrocode}
\NeedsTeXFormat{LaTeX2e}
\ProvidesPackage{fibeamer/theme/mu/%
  beamercolorthemefibeamer-mu-phil}[2016/05/06]
\mode<presentation>
%    \end{macrocode}
% This color theme uses the combination of yellow and shades of gray.  The
% |fibeamer@{dark,|\-|light}@background{Inner,|\-|Outer}| colors are used
% within the background canvas template, which is defined within the base
% color theme of the Masaryk University and which draws the gradient
% background of the frames.
%    \begin{macrocode}
  \definecolor{fibeamer@black}{HTML}{000000}
  \definecolor{fibeamer@white}{HTML}{FFFFFF}
  \definecolor{fibeamer@blue}{HTML}{0071B2}
  \colorlet{fibeamer@lightBlue}{fibeamer@blue!30!fibeamer@white}
  \colorlet{fibeamer@darkBlue}{fibeamer@blue!60!fibeamer@black}
  \definecolor{fibeamer@gray}{HTML}{999999}
  \definecolor{fibeamer@lightOrange}{HTML}{FFA25E}
  \colorlet{fibeamer@orange}{fibeamer@lightOrange!80!fibeamer@darkBlue}
%    \end{macrocode}
% \changes{v1.1.4:2}{2016/05/06}{Removed gradient backgrounds from
%   the color themes of the Masaryk University in Brno. [VN]}
%    \begin{macrocode}
  %% Background gradients
  \colorlet{fibeamer@dark@backgroundInner}{fibeamer@darkBlue}
  \colorlet{fibeamer@dark@backgroundOuter}{fibeamer@darkBlue}
  \colorlet{fibeamer@light@backgroundInner}{fibeamer@white}
  \colorlet{fibeamer@light@backgroundOuter}{fibeamer@white}
%    \end{macrocode}
% The |darkframes| environment switches the |\iffibeamer@darktrue|
% conditional on and sets a dark color theme.
%    \begin{macrocode}
  \renewenvironment{darkframes}{%
    \begingroup
      \fibeamer@darktrue
      %% Structures
      \setbeamercolor*{frametitle}{fg=fibeamer@lightBlue}
      \setbeamercolor*{framesubtitle}{fg=fibeamer@white}
      %% Text
      \setbeamercolor*{normal text}{fg=fibeamer@white, bg=fibeamer@blue}
      \setbeamercolor*{structure}{fg=fibeamer@white, bg=fibeamer@blue}
%    \end{macrocode}
% \changes{v1.1.0:7}{2016/01/12}{Added support for \cs{alert} to
%   the themes of the Masaryk University in Brno. [VN]}
% \changes{v1.1.4:5}{2016/05/06}{Unified the alert colors in the
%   color themes of the Masaryk University in Brno. [VN]}
%    \begin{macrocode}
      \setbeamercolor*{alerted text}{fg=fibeamer@lightOrange}
%    \end{macrocode}
% \changes{v1.1.4:3}{2016/05/06}{Added proper link coloring for the
%   color themes of the Masaryk University in Brno. [VN]}
% \changes{v1.1.6}{2017/04/23}{Added proper citation coloring for the
%   color themes of the Masaryk University in Brno. [VN]}
%    \begin{macrocode}
      %% Items, footnotes and links
      \setbeamercolor*{item}{fg=fibeamer@lightBlue}
      \setbeamercolor*{footnote mark}{fg=fibeamer@lightBlue}
      \hypersetup{urlcolor=fibeamer@lightBlue, citecolor=fibeamer@lightBlue}
      %% Blocks
      \setbeamercolor*{block title}{%
        fg=fibeamer@white, bg=fibeamer@blue!60!fibeamer@white}
      \setbeamercolor*{block title example}{%
        fg=fibeamer@white, bg=fibeamer@blue!60!fibeamer@white}
      \setbeamercolor*{block title alerted}{%
        fg=fibeamer@darkBlue, bg=fibeamer@lightOrange}
      \setbeamercolor*{block body}{%
        fg=fibeamer@blue, 
        bg=fibeamer@gray!15!fibeamer@white}
      \usebeamercolor*{normal text}
      % Code listings
      \lstset{%
        commentstyle=\color{green!30!fibeamer@white},
        keywordstyle=\color{fibeamer@lightBlue},
        stringstyle=\color{red!30!fibeamer@white}}
      }{%
    \endgroup}
%    \end{macrocode}
% Outside the |darkframes| environment, the light theme is used.
%    \begin{macrocode}
  %% Structures
  \setbeamercolor{frametitle}{fg=fibeamer@blue}
  \setbeamercolor{framesubtitle}{fg=fibeamer@black!75!fibeamer@white}
  %% Text
  \setbeamercolor{normal text}{fg=fibeamer@black, bg=fibeamer@white}
  \setbeamercolor{structure}{fg=fibeamer@black, bg=fibeamer@white}
%    \end{macrocode}
% \changes{v1.1.0:7}{2016/01/12}{Added support for \cs{alert} to
%   the themes of the Masaryk University in Brno. [VN]}
% \changes{v1.1.4:5}{2016/05/06}{Unified the alert colors in the
%   color themes of the Masaryk University in Brno. [VN]}
%    \begin{macrocode}
  \setbeamercolor{alerted text}{fg=red}
  \addtobeamertemplate{block begin}{%
    \iffibeamer@dark % alerted text in plain block at dark slides
      \setbeamercolor{alerted text}{fg=fibeamer@orange}%
    \fi}{}
%    \end{macrocode}
% \changes{v1.1.4:3}{2016/05/06}{Added proper link coloring for the
%   color themes of the Masaryk University in Brno. [VN]}
% \changes{v1.1.6}{2017/04/23}{Added proper citation coloring for the
%   color themes of the Masaryk University in Brno. [VN]}
%    \begin{macrocode}
  %% Items, footnotes and links
  \setbeamercolor*{item}{fg=fibeamer@blue}
  \setbeamercolor*{footnote mark}{fg=fibeamer@blue}
  \hypersetup{urlcolor=fibeamer@blue, citecolor=fibeamer@blue}
  %% Blocks
  \setbeamercolor{block title}{%
    fg=fibeamer@white, bg=fibeamer@blue}
  \setbeamercolor{block title example}{%
    fg=fibeamer@white, bg=fibeamer@blue}
  \setbeamercolor{block title alerted}{%
    fg=fibeamer@white, bg=red}
  \setbeamercolor{block body}{%
    fg=fibeamer@blue, bg=fibeamer@gray!20!fibeamer@white}
  %% Title
  \setbeamercolor{title}{fg=fibeamer@white, bg=fibeamer@blue}
  % Code listings
  \lstset{%
    basicstyle=\footnotesize\ttfamily,
    breakatwhitespace=false,
    breaklines=true,
    commentstyle=\color{green!60!fibeamer@black},
    extendedchars=true,
    keywordstyle=\color{fibeamer@blue},
    showspaces=false,
    showstringspaces=false,
    showtabs=false,
    stringstyle=\color{violet}}
\mode
<all>
%    \end{macrocode}

% \subsection{The Faculty of Education}
% % \file{theme/mu/beamercolorthemefibeamer-ped.sty}
% This is the color theme for presentations written at the
% Faculty of Education at the Masaryk University in Brno.
% This theme has no effect outside the presentation mode.
%    \begin{macrocode}
\NeedsTeXFormat{LaTeX2e}
\ProvidesPackage{fibeamer/theme/mu/%
  beamercolorthemefibeamer-mu-ped}[2016/01/12]
\mode<presentation>
%    \end{macrocode}
% This color theme uses the combination of orange and dark gray in light frames
% and the combination of gold and white in dark frames.  The
% |fibeamer@back|\-|groundInner| and |fibeamer@backgroundOuter| colors are used
% within the background canvas template, which is defined within the color
% theme of the Masaryk University and which draws the gradient background of
% dark frames.
%    \begin{macrocode}
  \definecolor{fibeamer@gold}{HTML}{CF7000}
  \definecolor{fibeamer@orange}{HTML}{FABE6E}
  \definecolor{fibeamer@darkOrange}{HTML}{F89C22}
  \definecolor{fibeamer@brightOrange}{HTML}{FF4000}
  \definecolor{fibeamer@gray}{HTML}{999999}
  \colorlet{fibeamer@backgroundInner}{fibeamer@gold}
  \colorlet{fibeamer@backgroundOuter}{fibeamer@gold!60!black}
%    \end{macrocode}
% The |darkframes| environment switches the |\iffibeamer@darktrue|
% conditional on and sets a dark color theme.
%    \begin{macrocode}
  \renewenvironment{darkframes}{%
    \begingroup
      \fibeamer@darktrue
      %% Structures
      \setbeamercolor*{frametitle}{fg=fibeamer@gold!30!white}
      \setbeamercolor*{framesubtitle}{fg=white}
      %% Text
      \setbeamercolor*{normal text}{fg=white, bg=fibeamer@gold}
      \setbeamercolor*{structure}{fg=white, bg=fibeamer@gold}
%    \end{macrocode}
% \changes{v1.1.0:7}{2016/01/12}{Added support for \cs{alert}. [VN]}
%    \begin{macrocode}
      \setbeamercolor*{alerted text}{%
        fg=white!40!fibeamer@brightOrange}
      %% Items
      \setbeamercolor*{item}{fg=fibeamer@gold!30!white}
      \setbeamercolor*{footnote mark}{fg=fibeamer@gold!30!white}
      %% Blocks
      \setbeamercolor*{block title}{
        use=structure, fg=white, bg=fibeamer@gold!60!white}
      \setbeamercolor*{block title example}{
        use=example text, fg=white, bg=fibeamer@gold!60!white}
      \setbeamercolor*{block title alerted}{
        use=alerted text, fg=white, bg=fibeamer@gold!90!white}
      \setbeamercolor*{block body}{
        fg=fibeamer@gold, use=block title,
        bg=fibeamer@gray!15!white}
      \usebeamercolor*{normal text}
      % Code listings
      \lstset{
        commentstyle=\color{green!30!white},
        keywordstyle=\color{blue!30!white},
        stringstyle=\color{fibeamer@gold!30!white}}
      }{%
    \endgroup}
%    \end{macrocode}
% Outside the |darkframes| environment, the light theme is used.
%    \begin{macrocode}
  %% Structures
  \setbeamercolor{frametitle}{fg=fibeamer@darkOrange}
  \setbeamercolor{framesubtitle}{fg=black!60!white}
  %% Text
  \setbeamercolor{normal text}{fg=black!75!white, bg=white}
  \setbeamercolor{structure}{fg=black!75!white, bg=white}
%    \end{macrocode}
% \changes{v1.1.0:7}{2016/01/12}{Added support for \cs{alert}. [VN]}
%    \begin{macrocode}
  \setbeamercolor{alerted text}{fg=fibeamer@brightOrange}
  \addtobeamertemplate{block begin}{%
    \iffibeamer@dark % alerted text in plain block at dark slides
      \setbeamercolor{alerted text}{%
        fg=white!20!fibeamer@brightOrange}%
    \else % alerted text in plain block at light slides
      \setbeamercolor{alerted text}{%
        fg=white!20!fibeamer@brightOrange}%
    \fi}{}
  %% Items
  \setbeamercolor{item}{fg=fibeamer@darkOrange}
  \setbeamercolor{footnote mark}{fg=fibeamer@darkOrange}
  %% Blocks
  \setbeamercolor{block title}{
    use=structure, fg=black!75!white, bg=fibeamer@orange}
  \setbeamercolor{block title example}{
    use=example text, fg=black!75!white, bg=fibeamer@orange}
  \setbeamercolor{block title alerted}{
    use=alerted text, fg=white, bg=black!75!white}
  \setbeamercolor{block body}{
    fg=fibeamer@orange, use=block title, bg=black!75!white}
  %% Title
  \setbeamercolor{title}{fg=white, bg=fibeamer@gold}
  \setbeamercolor{title}{use=structure}
  % Code listings
  \lstset{
    basicstyle=\footnotesize\ttfamily,
    breakatwhitespace=false,
    breaklines=true,
    commentstyle=\color{green!60!black},
    extendedchars=true,
    keywordstyle=\color{blue},
    showspaces=false,
    showstringspaces=false,
    showtabs=false,
    stringstyle=\color{violet}}
\mode
<all>
%    \end{macrocode}

% \subsection{The Faculty of Social Studies}
% % \file{theme/mu/beamercolorthemefibeamer-fss.sty}
% This is the color theme for presentations written at the Faculty
% of Social Studies at the Masaryk University in Brno.  This theme
% has no effect outside the presentation mode.
%    \begin{macrocode}
\NeedsTeXFormat{LaTeX2e}
\ProvidesPackage{fibeamer/theme/mu/%
  beamercolorthemefibeamer-mu-fss}[2016/01/14]
\mode<presentation>
%    \end{macrocode}
% This color theme uses the combination of cyan, gray and
% white.  The |fibeamer@back|\-|groundInner| and
% |fibeamer@backgroundOuter| colors are used within the background
% canvas template, which is defined within the color theme of the
% Masaryk University and which draws the gradient background of
% dark frames.
%    \begin{macrocode}
  \definecolor{fibeamer@black}{HTML}{000000}
  \definecolor{fibeamer@white}{HTML}{FFFFFF}
  \definecolor{fibeamer@cyan}{HTML}{00796E}
  \definecolor{fibeamer@brightCyan}{HTML}{06c696}
  \definecolor{fibeamer@gray}{HTML}{999999}
%    \end{macrocode}
% \changes{v1.1.4:2}{2016/05/06}{Removed gradient backgrounds from the color
% themes for the Faculty of Informatics at the Masaryk University in Brno.
% [VN]}
%    \begin{macrocode}
  %% Background gradients
  \colorlet{fibeamer@dark@backgroundInner}{fibeamer@cyan!70!fibeamer@black}
  \colorlet{fibeamer@dark@backgroundOuter}{fibeamer@cyan!70!fibeamer@black}
  \colorlet{fibeamer@light@backgroundInner}{fibeamer@white}
  \colorlet{fibeamer@light@backgroundOuter}{fibeamer@white}
%    \end{macrocode}
% The |darkframes| environment switches the |\iffibeamer@darktrue|
% conditional on and sets a dark color theme.
%    \begin{macrocode}
  \renewenvironment{darkframes}{%
    \begingroup
      \fibeamer@darktrue
      %% Structures
      \setbeamercolor*{frametitle}{fg=fibeamer@cyan!30!fibeamer@white}
      \setbeamercolor*{framesubtitle}{fg=fibeamer@white}
      %% Text
      \setbeamercolor*{normal text}{fg=fibeamer@white, bg=fibeamer@cyan}
      \setbeamercolor*{structure}{fg=fibeamer@white, bg=fibeamer@cyan}
%    \end{macrocode}
% \changes{v1.1.0:7}{2016/01/12}{Added support for \cs{alert} to
%   the themes of the Masaryk University in Brno. [VN]}
%    \begin{macrocode}
      \setbeamercolor*{alerted text}{fg=fibeamer@brightCyan!80!fibeamer@white}
      %% Items
      \setbeamercolor*{item}{fg=fibeamer@cyan!30!fibeamer@white}
      \setbeamercolor*{footnote mark}{fg=fibeamer@cyan!30!fibeamer@white}
      %% Blocks
      \setbeamercolor*{block title}{%
        fg=fibeamer@white, bg=fibeamer@cyan!60!fibeamer@white}
      \setbeamercolor*{block title example}{%
        fg=fibeamer@white, bg=fibeamer@cyan!60!fibeamer@white}
      \setbeamercolor*{block title alerted}{%
        fg=fibeamer@white, bg=fibeamer@cyan}
      \setbeamercolor*{block body}{%
        fg=fibeamer@cyan, 
        bg=fibeamer@gray!15!fibeamer@white}
      \usebeamercolor*{normal text}
      % Code listings
      \lstset{%
        commentstyle=\color{green!30!fibeamer@white},
        keywordstyle=\color{blue!30!fibeamer@white},
        stringstyle=\color{red!30!fibeamer@white}}
      }{%
    \endgroup}
%    \end{macrocode}
% Outside the |darkframes| environment, the light theme is used.
%    \begin{macrocode}
  %% Structures
  \setbeamercolor{frametitle}{fg=fibeamer@cyan}
  \setbeamercolor{framesubtitle}{fg=fibeamer@black!75!fibeamer@white}
  %% Text
  \setbeamercolor{normal text}{fg=fibeamer@black, bg=fibeamer@white}
  \setbeamercolor{structure}{fg=fibeamer@black, bg=fibeamer@white}
%    \end{macrocode}
% \changes{v1.1.0:7}{2016/01/12}{Added support for \cs{alert} to
%   the themes of the Masaryk University in Brno. [VN]}
%    \begin{macrocode}
  \setbeamercolor{alerted text}{fg=fibeamer@brightCyan}
  \addtobeamertemplate{block begin}{%
    \iffibeamer@dark % alerted text in plain block at dark slides
      \setbeamercolor{alerted text}{fg=fibeamer@brightCyan}%
    \else % alerted text in plain block at light slides
      \setbeamercolor{alerted text}{fg=fibeamer@brightCyan!80!fibeamer@white}%
    \fi}{}
  %% Items
  \setbeamercolor{item}{fg=fibeamer@cyan}
  \setbeamercolor{footnote mark}{fg=fibeamer@cyan}
  %% Blocks
  \setbeamercolor{block title}{%
    fg=fibeamer@white, bg=fibeamer@cyan!50!fibeamer@white}
  \setbeamercolor{block title example}{%
    fg=fibeamer@white, bg=fibeamer@cyan!50!fibeamer@white}
  \setbeamercolor{block title alerted}{%
    fg=fibeamer@white, bg=fibeamer@cyan}
  \setbeamercolor{block body}{%
    fg=fibeamer@cyan, bg=fibeamer@gray!20!fibeamer@white}
  %% Title
  \setbeamercolor{title}{fg=fibeamer@white, bg=fibeamer@cyan}
  % Code listings
  \lstset{%
    basicstyle=\footnotesize\ttfamily,
    breakatwhitespace=false,
    breaklines=true,
    commentstyle=\color{green!60!fibeamer@black},
    extendedchars=true,
    keywordstyle=\color{blue},
    showspaces=false,
    showstringspaces=false,
    showtabs=false,
    stringstyle=\color{violet}}
\mode
<all>
%    \end{macrocode}

% \subsection{The Faculty of Law}
% % \file{theme/mu/beamercolorthemefibeamer-law.sty}
% This is the color theme for presentations written at the Faculty
% of Law at the Masaryk University in Brno. This theme has no
% effect outside the presentation mode.
%    \begin{macrocode}
\NeedsTeXFormat{LaTeX2e}
\ProvidesPackage{fibeamer/theme/mu/%
  beamercolorthemefibeamer-mu-law}[2015/08/26]
\mode<presentation>
%    \end{macrocode}
% This color theme uses the combination of violet, gray and white.
% The |fibeamer@back|\-|groundInner| and |fibeamer@backgroundOuter|
% colors are used within the background canvas template, which is
% defined within the color theme of the Masaryk University and
% which draws the gradient background of dark frames.
%    \begin{macrocode}
  \definecolor{fibeamer@violet}{HTML}{660099}
  \definecolor{fibeamer@gray}{HTML}{999999}
  \colorlet{fibeamer@backgroundInner}{fibeamer@violet}
  \colorlet{fibeamer@backgroundOuter}{fibeamer@violet!60!black}
%    \end{macrocode}
% The |darkframes| environment switches the |\iffibeamer@darktrue|
% conditional on and sets a dark color theme.
%    \begin{macrocode}
  \renewenvironment{darkframes}{%
    \begingroup
      \fibeamer@darktrue
      %% Structures
      \setbeamercolor*{structure}{fg=white, bg=fibeamer@violet}
      \setbeamercolor*{frametitle}{fg=fibeamer@violet!25!white}
      \setbeamercolor*{framesubtitle}{fg=white}
      %% Text
      \setbeamercolor*{normal text}{fg=white, bg=white}
      %% Items
      \setbeamercolor*{item}{fg=fibeamer@violet!25!white}
      \setbeamercolor{footnote mark}{fg=fibeamer@violet!25!white}
      %% Blocks
      \setbeamercolor*{block title}{
        use=structure, fg=white, bg=fibeamer@violet!60!white}
      \setbeamercolor*{block title example}{
        use=example text, fg=white, bg=fibeamer@violet!60!white}
      \setbeamercolor*{block title alerted}{
        use=alerted text, fg=white, bg=fibeamer@violet!90!white}
      \setbeamercolor*{block body}{
        fg=fibeamer@violet, use=block title,
        bg=fibeamer@gray!15!white}
      \usebeamercolor*{normal text}
      % Code listings
      \lstset{
        commentstyle=\color{green!25!white},
        keywordstyle=\color{blue!25!white},
        stringstyle=\color{fibeamer@violet!25!white}}
      }{%
    \endgroup}
%    \end{macrocode}
% Outside the |darkframes| environment, the light theme is used.
%    \begin{macrocode}
  %% Structures
  \setbeamercolor{structure}{fg=black!75!white, bg=white}
  \setbeamercolor{frametitle}{fg=fibeamer@violet}
  \setbeamercolor{framesubtitle}{fg=black!75!white}
  %% Text
  \setbeamercolor{normal text}{fg=black}
  %% Items
  \setbeamercolor{item}{fg=fibeamer@violet}
  \setbeamercolor{footnote mark}{fg=fibeamer@violet}
  %% Blocks
  \setbeamercolor{block title}{
    use=structure, fg=white, bg=fibeamer@violet!50!white}
  \setbeamercolor{block title example}{
    use=example text, fg=white, bg=fibeamer@violet!50!white}
  \setbeamercolor{block title alerted}{
    use=alerted text, fg=white, bg=fibeamer@violet}
  \setbeamercolor{block body}{
    fg=fibeamer@violet, use=block title, bg=fibeamer@gray!20!white}
  %% Title
  \setbeamercolor{title}{fg=white, bg=fibeamer@violet}
  \setbeamercolor{title}{use=structure}
  % Code listings
  \lstset{
    basicstyle=\footnotesize\ttfamily,
    breakatwhitespace=false,
    breaklines=true,
    commentstyle=\color{green!60!black},
    extendedchars=true,
    keywordstyle=\color{blue},
    showspaces=false,
    showstringspaces=false,
    showtabs=false,
    stringstyle=\color{violet}}
\mode
<all>
%    \end{macrocode}

% \subsection{The Faculty of Economics and Administration}
% % \file{theme/mu/beamercolorthemefibeamer-econ.sty}
% This is the color theme for presentations written at the Faculty
% of Economics and Administration at the Masaryk University in
% Brno. This theme has no effect outside the presentation mode.
%    \begin{macrocode}
\NeedsTeXFormat{LaTeX2e}
\ProvidesPackage{fibeamer/theme/mu/%
  beamercolorthemefibeamer-mu-econ}[2016/05/06]
\mode<presentation>
%    \end{macrocode}
% This color theme uses the combination of red-brown, gray and
% white.  The |fibeamer@back|\-|groundInner| and
% |fibeamer@backgroundOuter| colors are used within the background
% canvas template, which is defined within the color theme of the
% Masaryk University and which draws the gradient background of
% dark frames.
%    \begin{macrocode}
  \definecolor{fibeamer@black}{HTML}{2B2E34}
  \definecolor{fibeamer@white}{HTML}{FFFFFF}
  \definecolor{fibeamer@brown}{HTML}{892840}
  \colorlet{fibeamer@lightBrown}{fibeamer@brown!30!fibeamer@white}
  \definecolor{fibeamer@gray}{HTML}{999999}
%    \end{macrocode}
% \changes{v1.1.4:2}{2016/05/06}{Removed gradient backgrounds from the color
%   themes of the Masaryk University in Brno. [VN]}
%    \begin{macrocode}
  %% Background gradients
  \colorlet{fibeamer@dark@backgroundInner}{fibeamer@brown!60!fibeamer@black}
  \colorlet{fibeamer@dark@backgroundOuter}{fibeamer@brown!60!fibeamer@black}
  \colorlet{fibeamer@light@backgroundInner}{fibeamer@white}
  \colorlet{fibeamer@light@backgroundOuter}{fibeamer@white}
%    \end{macrocode}
% The |darkframes| environment switches the |\iffibeamer@darktrue|
% conditional on and sets a dark color theme.
%    \begin{macrocode}
  \renewenvironment{darkframes}{%
    \begingroup
      \fibeamer@darktrue
      %% Structures
      \setbeamercolor*{frametitle}{fg=fibeamer@brown!30!fibeamer@white}
      \setbeamercolor*{framesubtitle}{fg=fibeamer@white}
      %% Text
      \setbeamercolor*{normal text}{fg=fibeamer@white, bg=fibeamer@brown}
      \setbeamercolor*{structure}{fg=fibeamer@white, bg=fibeamer@brown}
%    \end{macrocode}
% \changes{v1.1.0:7}{2016/01/12}{Added support for \cs{alert} to
%   the themes of the Masaryk University in Brno. [VN]}
%    \begin{macrocode}
      \setbeamercolor*{alerted text}{fg=red!60!fibeamer@white}
%    \end{macrocode}
% \changes{v1.1.4:3}{2016/05/06}{Added proper link coloring for the
%   color themes of the Masaryk University in Brno. [VN]}
%    \begin{macrocode}
      %% Items, footnotes and links
      \setbeamercolor*{item}{fg=fibeamer@lightBrown}
      \setbeamercolor*{footnote mark}{fg=fibeamer@lightBrown}
      \hypersetup{urlcolor=fibeamer@lightBrown}
      %% Blocks
      \setbeamercolor*{block title}{%
        fg=fibeamer@white, bg=fibeamer@brown!60!fibeamer@white}
      \setbeamercolor*{block title example}{%
        fg=fibeamer@white, bg=fibeamer@lightBrown}
      \setbeamercolor*{block title alerted}{%
        fg=fibeamer@white, bg=fibeamer@brown!90!fibeamer@white}
      \setbeamercolor*{block body}{%
        fg=fibeamer@brown,
        bg=fibeamer@gray!15!fibeamer@white}
      \usebeamercolor*{normal text}
      % Code listings
      \lstset{%
        commentstyle=\color{green!30!fibeamer@white},
        keywordstyle=\color{blue!30!fibeamer@white},
        stringstyle=\color{fibeamer@brown!30!fibeamer@white}}
      }{%
    \endgroup}
%    \end{macrocode}
% Outside the |darkframes| environment, the light theme is used.
%    \begin{macrocode}
  %% Structures
  \setbeamercolor{frametitle}{fg=fibeamer@brown}
  \setbeamercolor{framesubtitle}{fg=fibeamer@black!75!fibeamer@white}
  %% Text
  \setbeamercolor{normal text}{fg=fibeamer@black, bg=fibeamer@white}
  \setbeamercolor{structure}{fg=fibeamer@black, bg=fibeamer@white}
%    \end{macrocode}
% \changes{v1.1.0:7}{2016/01/12}{Added support for \cs{alert} to
%   the themes of the Masaryk University in Brno. [VN]}
%    \begin{macrocode}
  \setbeamercolor{alerted text}{fg=red}
  \addtobeamertemplate{block begin}{%
    \iffibeamer@dark % alerted text in plain block at dark slides
      \setbeamercolor{alerted text}{fg=red}%
    \fi}{}
%    \end{macrocode}
% \changes{v1.1.4:3}{2016/05/06}{Added proper link coloring for the
%   color themes of the Masaryk University in Brno. [VN]}
%    \begin{macrocode}
  %% Items, footnotes and links
  \setbeamercolor{item}{fg=fibeamer@brown}
  \setbeamercolor{footnote mark}{fg=fibeamer@brown}
  \hypersetup{urlcolor=fibeamer@brown}
  %% Blocks
  \setbeamercolor{block title}{%
    fg=fibeamer@white, bg=fibeamer@brown!50!fibeamer@white}
  \setbeamercolor{block title example}{%
    fg=fibeamer@white, bg=fibeamer@brown!50!fibeamer@white}
  \setbeamercolor{block title alerted}{%
    fg=fibeamer@white, bg=fibeamer@brown}
  \setbeamercolor{block body}{%
    fg=fibeamer@brown, bg=fibeamer@gray!20!fibeamer@white}
  %% Title
  \setbeamercolor{title}{fg=fibeamer@white, bg=fibeamer@brown}
  % Code listings
  \lstset{%
    basicstyle=\footnotesize\ttfamily,
    breakatwhitespace=false,
    breaklines=true,
    commentstyle=\color{green!60!fibeamer@black},
    extendedchars=true,
    keywordstyle=\color{blue},
    showspaces=false,
    showstringspaces=false,
    showtabs=false,
    stringstyle=\color{violet}}
\mode
<all>
%    \end{macrocode}

% \subsection{The Faculty of Medicine}
% % \file{theme/mu/beamercolorthemefibeamer-med.sty}
% This is the color theme for presentations written at the Faculty
% of Medicine at the Masaryk University in Brno. This theme has no
% effect outside the presentation mode.
%    \begin{macrocode}
\NeedsTeXFormat{LaTeX2e}
\ProvidesPackage{fibeamer/theme/mu/%
  beamercolorthemefibeamer-mu-med}[2016/05/06]
\mode<presentation>
%    \end{macrocode}
% This color theme uses the combination of red, gray and
% white. The |fibeamer@back|\-|groundInner| and
% |fibeamer@backgroundOuter| colors are used within the background
% canvas template, which is defined within the color theme of the
% Masaryk University and which draws the gradient background of
% dark frames.
%    \begin{macrocode}
  \definecolor{fibeamer@black}{HTML}{000000}
  \definecolor{fibeamer@white}{HTML}{FFFFFF}
  \definecolor{fibeamer@red}{HTML}{c82600}
  \definecolor{fibeamer@brightRed}{HTML}{FF350A}
  \definecolor{fibeamer@gray}{HTML}{999999}
%    \end{macrocode}
% \changes{v1.1.4:2}{2016/05/06}{Removed gradient backgrounds from the color
% themes for the Faculty of Informatics at the Masaryk University in Brno.
% [VN]}
%    \begin{macrocode}
  %% Background gradients
  \colorlet{fibeamer@dark@backgroundInner}{fibeamer@red!60!fibeamer@black}
  \colorlet{fibeamer@dark@backgroundOuter}{fibeamer@red!60!fibeamer@black}
  \colorlet{fibeamer@light@backgroundInner}{fibeamer@white}
  \colorlet{fibeamer@light@backgroundOuter}{fibeamer@white}
%    \end{macrocode}
% The |darkframes| environment switches the |\iffibeamer@darktrue|
% conditional on and sets a dark color theme.
%    \begin{macrocode}
  \renewenvironment{darkframes}{%
    \begingroup
      \fibeamer@darktrue
      %% Structures
      \setbeamercolor*{frametitle}{fg=fibeamer@red!30!fibeamer@white}
      \setbeamercolor*{framesubtitle}{fg=fibeamer@white}
      %% Text
      \setbeamercolor*{normal text}{fg=fibeamer@white, bg=fibeamer@red}
      \setbeamercolor*{structure}{fg=fibeamer@white, bg=fibeamer@red}
%    \end{macrocode}
% \changes{v1.1.0:7}{2016/01/12}{Added support for \cs{alert} to
%   the themes of the Masaryk University in Brno. [VN]}
%    \begin{macrocode}
      \setbeamercolor*{alerted text}{%
        fg=fibeamer@brightRed!60!fibeamer@white!80!yellow}
      %% Items
      \setbeamercolor*{item}{fg=fibeamer@red!30!fibeamer@white}
      \setbeamercolor*{footnote mark}{fg=fibeamer@red!30!fibeamer@white}
      %% Blocks
      \setbeamercolor*{block title}{%
        fg=fibeamer@white, bg=fibeamer@red!60!fibeamer@white}
      \setbeamercolor*{block title example}{%
        fg=fibeamer@white, bg=fibeamer@red!60!fibeamer@white}
      \setbeamercolor*{block title alerted}{%
        fg=fibeamer@white, bg=fibeamer@red!90!fibeamer@white}
      \setbeamercolor*{block body}{%
        fg=fibeamer@red, 
        bg=fibeamer@gray!15!fibeamer@white}
      \usebeamercolor*{normal text}
      % Code listings
      \lstset{%
        commentstyle=\color{green!30!fibeamer@white},
        keywordstyle=\color{blue!30!fibeamer@white},
        stringstyle=\color{fibeamer@red!30!fibeamer@white}}
      }{%
    \endgroup}
%    \end{macrocode}
% Outside the |darkframes| environment, the light theme is used.
%    \begin{macrocode}
  %% Structures
  \setbeamercolor{frametitle}{fg=fibeamer@red}
  \setbeamercolor{framesubtitle}{fg=fibeamer@black!75!fibeamer@white}
  %% Text
  \setbeamercolor{normal text}{fg=fibeamer@black, bg=fibeamer@white}
  \setbeamercolor{structure}{fg=fibeamer@black, bg=fibeamer@white}
%    \end{macrocode}
% \changes{v1.1.0:7}{2016/01/12}{Added support for \cs{alert} to
%   the themes of the Masaryk University in Brno. [VN]}
%    \begin{macrocode}
  \setbeamercolor{alerted text}{fg=fibeamer@brightRed}
  \addtobeamertemplate{block begin}{%
    \iffibeamer@dark % alerted text in plain block at dark slides
      \setbeamercolor{alerted text}{%
        fg=fibeamer@brightRed!80!fibeamer@white!80!yellow}%
    \else % alerted text in plain block at light slides
      \setbeamercolor{alerted text}{%
        fg=fibeamer@brightRed!80!fibeamer@white}%
    \fi}{}
  %% Items
  \setbeamercolor{item}{fg=fibeamer@red}
  \setbeamercolor{footnote mark}{fg=fibeamer@red}
  %% Blocks
  \setbeamercolor{block title}{%
    fg=fibeamer@white, bg=fibeamer@red!50!fibeamer@white}
  \setbeamercolor{block title example}{%
    fg=fibeamer@white, bg=fibeamer@red!50!fibeamer@white}
  \setbeamercolor{block title alerted}{%
    fg=fibeamer@white, bg=fibeamer@red}
  \setbeamercolor{block body}{%
    fg=fibeamer@red!90!fibeamer@black, 
    bg=fibeamer@gray!20!fibeamer@white}
  %% Title
  \setbeamercolor{title}{fg=fibeamer@white, bg=fibeamer@red}
  % Code listings
  \lstset{%
    basicstyle=\footnotesize\ttfamily,
    breakatwhitespace=false,
    breaklines=true,
    commentstyle=\color{green!60!fibeamer@black},
    extendedchars=true,
    keywordstyle=\color{blue},
    showspaces=false,
    showstringspaces=false,
    showtabs=false,
    stringstyle=\color{violet}}
\mode
<all>
%    \end{macrocode}

% \subsection{The Faculty of Sports Studies}
% % \file{theme/mu/beamercolorthemefibeamer-fsps.sty}
% This is the color theme for presentations written at the Faculty
% of Sports Studies at the Masaryk University in Brno. This theme
% has no effect outside the presentation mode.
%    \begin{macrocode}
\NeedsTeXFormat{LaTeX2e}
\ProvidesPackage{fibeamer/theme/mu/%
  beamercolorthemefibeamer-mu-fsps}[2016/01/11]
\mode<presentation>
%    \end{macrocode}
% This color theme uses the combination of dark blue, gray and
% white. The |fibeamer@back|\-|groundInner| and
% |fibeamer@backgroundOuter| colors are used within the background
% canvas template, which is defined within the color theme of the
% Masaryk University and which draws the gradient background of
% dark frames.
%    \begin{macrocode}
  \definecolor{fibeamer@blue}{HTML}{1B458F}
  \definecolor{fibeamer@gray}{HTML}{999999}
  \colorlet{fibeamer@backgroundInner}{fibeamer@blue}
  \colorlet{fibeamer@backgroundOuter}{fibeamer@blue!60!black}
%    \end{macrocode}
% The |darkframes| environment switches the |\iffibeamer@darktrue|
% conditional on and sets a dark color theme.
%    \begin{macrocode}
  \renewenvironment{darkframes}{%
    \begingroup
      \fibeamer@darktrue
      %% Structures
      \setbeamercolor*{structure}{fg=white, bg=fibeamer@blue}
      \setbeamercolor*{frametitle}{fg=fibeamer@blue!25!white}
      \setbeamercolor*{framesubtitle}{fg=white}
      %% Text
      \setbeamercolor*{normal text}{fg=white, bg=fibeamer@blue}
      %% Items
      \setbeamercolor*{item}{fg=fibeamer@blue!25!white}
      \setbeamercolor{footnote mark}{fg=fibeamer@blue!25!white}
      %% Blocks
      \setbeamercolor*{block title}{
        use=structure, fg=white, bg=fibeamer@blue!60!white}
      \setbeamercolor*{block title example}{
        use=example text, fg=white, bg=fibeamer@blue!60!white}
      \setbeamercolor*{block title alerted}{
        use=alerted text, fg=white, bg=fibeamer@blue!95!white}
      \setbeamercolor*{block body}{
        fg=fibeamer@blue, use=block title,
        bg=fibeamer@gray!15!white}
      \usebeamercolor*{normal text}
      % Code listings
      \lstset{
        commentstyle=\color{green!25!white},
        keywordstyle=\color{blue!25!white},
        stringstyle=\color{red!25!white}}
      }{%
    \endgroup}
%    \end{macrocode}
% Outside the |darkframes| environment, the light theme is used.
%    \begin{macrocode}
  %% Structures
  \setbeamercolor{structure}{fg=black!75!white, bg=white}
  \setbeamercolor{frametitle}{fg=fibeamer@blue}
  \setbeamercolor{framesubtitle}{fg=black!75!white}
  %% Text
  \setbeamercolor{normal text}{fg=black}
  %% Items
  \setbeamercolor{item}{fg=fibeamer@blue}
  \setbeamercolor{footnote mark}{fg=fibeamer@blue}
  %% Blocks
  \setbeamercolor{block title}{
    use=structure, fg=white, bg=fibeamer@blue!50!white}
  \setbeamercolor{block title example}{
    use=example text, fg=white, bg=fibeamer@blue!50!white}
  \setbeamercolor{block title alerted}{
    use=alerted text, fg=white, bg=fibeamer@blue}
  \setbeamercolor{block body}{
    fg=fibeamer@blue, use=block title, bg=fibeamer@gray!20!white}
  %% Title
  \setbeamercolor{title}{fg=white, bg=fibeamer@blue}
  \setbeamercolor{title}{use=structure}
  % Code listings
  \lstset{
    basicstyle=\footnotesize\ttfamily,
    breakatwhitespace=false,
    breaklines=true,
    commentstyle=\color{green!60!black},
    extendedchars=true,
    keywordstyle=\color{blue},
    showspaces=false,
    showstringspaces=false,
    showtabs=false,
    stringstyle=\color{violet}}
\mode
<all>
%    \end{macrocode}

% \iffalse
%</class>
% \fi
